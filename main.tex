\documentclass[12pt]{article}
\usepackage{graphicx} % Required for inserting images
\usepackage[letter,margin=1in]{geometry} % Adjust margins if needed
\usepackage[utf8]{inputenc}
\usepackage{url}
\usepackage[
backend=biber,
style=nature,
]{biblatex}
\addbibresource{references.bib}

\begin{document}

\begin{center}
    \Large{\textbf{A Geospatial and Temporal Analysis of Lead Pollution From Piston Aircrafts Near Kelowna International Airport Using ICP-MS with ADS-B Data}}\\
    \vspace{1cm}
    \large{Wenqi Guo}\\
    \vspace{0.5cm}
    \textit{Project Duration: 4 months}\\
    \vspace{0.5cm}
    \textit{Funding Agency: UBC and Interior Health}\\
    \vspace{0.5cm}
    \textbf{Amount Required: \$1000}\\
    \vspace{1cm}
    \vspace{1cm}
    \today
\end{center}
\newpage

\section{Introduction and Background}
Lead has been known as a type of toxin and there is no known safe amount of lead: even a small amount of lead could lead to health issues, such as decreased IQ in children, high blood pressure, cardiovascular problems and kidney damage. \cite{world_health_organization_lead_2023} Although leaded gasoline has been banned for on-road vehicles in the world, it is still used in many piston-engine powered small general aviation aircraft as 100LL (100 Low Lead), which contain tetraethyllead (TEL) and has a lead concentration of about 0.56g/L. 
\cite{noauthor_safety_2021} Previous study has shown that children near general aviation airports have elevated blood lead levels (BLL). \cite{miranda_geospatial_2011} \cite{zahran_leaded_2023} \cite{mills_lead_2022} \cite{zahran_effect_2017} Additionally, the United States Environmental Protection Agency has very recently issued a determination that lead-emission from position engine aircraft cause or contribute to air pollution, giving reasonable anticipation that it may endanger public health and welfare.  \cite{us_epa_epa_2023} 
% duzihlkouke rewrites

Kelowna International Airport (CYLW) serves both commercial and general aviation airplanes. Although its location is remote from Kelowna downtown, it is not distanced from some residential neighbourhoods such as Glenmore Highland and Rutland (about 5km and 4km to the end of CYLW runway, respectively). Given Kelowna's growing population and potential growth in residential areas, the close proximity of the airport is a valid public health concern. Additionally, the end of its runway is only about 1km from the University of British Columbia Okanagan Campus (UBCO), which is a growing educational and research institution with many students and staff, and there are about 2000 students who live on campus\footnote{\url{https://ok.ubc.ca/about/facts-and-figures/}}. This also raised concerns about possible lead contamination around the campus.

In this study, we will collect the topsoil samples around the airport and analyze their lead level using ICP-MS. We will also collect the Automatic Dependent Surveillance–Broadcast (ADS-B) data from the aircraft near CYLW from ADSB Exchange\cite{adsbexchange}, which provides the tracking data of an aircraft's real-time information like location and altitude. We will then analyze the geospatial and temporal correlation between the ADS-B data and lead level and determine if the topsoil lead level is higher under the flight paths or when flight activities are more frequent. 
% can we also do airborn bisai biasiazuishbashouzhikunyunex
\section{Statement of Purpose}
Although leaded gasoline has been banned worldwide for on-road vehicles, it is still commonly used for small piston-engine powered general aviation (GA) aircraft. (The fuel commercial jets use is not leaded.) Previous study has shown that children near GA airports have elevated blood lead level.\cite{miranda_geospatial_2011} \cite{zahran_leaded_2023} \cite{mills_lead_2022} \cite{zahran_effect_2017} Kelowna International Airport is a busy airport that serves both GA and commercial aircraft. With the growing population in Kelowna and the UBCO campus, the lead emission from GA aircraft in this area is a public health concern. In this study, we aim to analyze the topsoil lead level around the Kelowna International Airport using ICP-MS (an elemental analysis technique). We will incorporate it with the ADS-B data near the airport, which will provide the location and altitude information of the aircraft. This allows us to find the geospatial and temporal relations between the aircraft's activities and topsoil lead level.
\section{Methods}
\subsection{Sample Collection And Analysis}
\subsection{ADS-B Data Collection}
\subsection{Data Analysis}
\newpage
\printbibliography
\end{document}
