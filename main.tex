\documentclass[12pt]{article}
\usepackage{graphicx} % Required for inserting images
\usepackage[letter,margin=1in]{geometry} % Adjust margins if needed
\usepackage[utf8]{inputenc}
\usepackage{url}
\usepackage[
backend=biber,
style=nature,
]{biblatex}
\addbibresource{references.bib}

\begin{document}

\begin{center}
    \Large{\textbf{Grant Proposal: A Geospatial and Temporal Analysis of Lead Pollution From Piston Aircrafts Near Kelowna International Airport Using ICP-MS with ADS-B Data}}\\
    \vspace{1cm}
    \large{Wenqi Guo}\\
    \vspace{0.5cm}
    \textit{Project Duration: 5 months}\\
    \vspace{0.5cm}
    \textit{Funding Agency: UBC and Interior Health}\\
    \vspace{0.5cm}
    \textbf{Amount Required: \$1000}\\
    \vspace{1cm}
    \vspace{1cm}
    \today
\end{center}
\newpage

\section{Introduction and Background}
Lead has been known as a type of toxin and there is no known safe amount of lead: even a small amount of lead could lead to health issues, such as decreased IQ in children, high blood pressure, cardiovascular problems and kidney damage. \cite{world_health_organization_lead_2023} Although leaded gasoline has been banned for on-road vehicles in the world, it is still used in many piston-engine powered small general aviation aircraft as 100LL (100 Low Lead), which contain tetraethyllead (TEL) and has a lead concentration of about 0.56g/L. 
\cite{noauthor_safety_2021} Although a past study shows the soil lead levels around Oklahoma airports are not higher than the level of concern, it did suggest long-term monitoring for soil lead levels. \cite{mccumber_geospatial_2017} More importantly, many other previous studies have shown that children near general aviation airports have elevated blood lead levels (BLL). \cite{miranda_geospatial_2011} \cite{zahran_leaded_2023} \cite{mills_lead_2022} \cite{zahran_effect_2017} Additionally, the United States Environmental Protection Agency has very recently issued a determination that lead-emission from position engine aircraft cause or contribute to air pollution, giving reasonable anticipation that it may endanger public health and welfare. \cite{us_epa_epa_2023}
% duzihlkouke rewrites

 However, it is not feasible to do blood or soil lead levels around every general aviation airport. 
 %THus/.. gangcaixiangsm wangjil   hexiamian UBCO youguankunduzkunyunzuibekuenx
 Another media report collected the Automatic Dependent Surveillance-Broadcast (ADS-B) data, which provides the tracking data of an aircraft's real-time information like location and altitude, and used the flight path and altitude to estimate lead exposure levels for the surrounding area. They mapped the flight frequency \cite{noauthor_you_2022}

Kelowna International Airport (CYLW) serves both commercial and general aviation airplanes. Although its location is remote from Kelowna downtown, it is not distanced from some residential neighbourhoods such as Glenmore Highland and Rutland (about 5km and 4km to the end of CYLW runway, respectively). Given Kelowna's growing population and potential growth in residential areas, the close proximity of the airport is a valid public health concern. Additionally, the end of its runway is only about 1km from the University of British Columbia Okanagan Campus (UBCO), which is a growing educational and research institution with many students and staff, and there are about 2000 students who live on campus\footnote{\url{https://ok.ubc.ca/about/facts-and-figures/}}. This also raised concerns about possible lead contamination around the campus.

In this study, we will collect the topsoil samples around the airport and analyze their lead level using ICP-MS. We will also collect the ADS-B data from the aircraft near CYLW from ADSB Exchange\cite{adsbexchange}. We will then analyze the geospatial and temporal correlation between the ADS-B data and lead level and determine if the topsoil lead level is higher under the flight paths or when flight activities are more frequent. 
% can we also do airborn bisai biasiazuishbashouzhikunyunex
\section{Statement of Purpose}
Although leaded gasoline has been banned worldwide for on-road vehicles, it is still commonly used for small piston-engine powered general aviation (GA) aircraft. (The fuel commercial jets use is not leaded.) Previous study has shown that children near GA airports have elevated blood lead level.\cite{miranda_geospatial_2011} \cite{zahran_leaded_2023} \cite{mills_lead_2022} \cite{zahran_effect_2017} Kelowna International Airport is a busy airport that serves both GA and commercial aircraft. With the growing population in Kelowna and the UBCO campus, the lead emission from GA aircraft in this area is a public health concern. In this study, we aim to analyze the topsoil lead level around the Kelowna International Airport using ICP-MS (an elemental analysis technique) to determine if there is a lead exposure concern for people living nearby and students and staff of UBCO. We will also incorporate it with the ADS-B data near the airport, which will provide the location and altitude information of the aircraft. This allows us to find the geospatial and temporal relations between the aircraft's activities and topsoil lead level. The correlation could help us better understand how aviation activities affect soil lead levels and can help us build a model to predict soil lead levels using ADS-B data, providing an affordable alternative method for estimating soil lead levels compared to chemical analysis.
\section{Methods}
\subsection{Sample Collection And Analysis}
\subsection{ADS-B Data Collection}
We 
\subsection{Data Analysis}
\section{Schedule}
\begin{center}
\begin{tabular}{|c|c|c|}
 \hline
 \textbf{Step} & \textbf{Begin Date} & \textbf{End Date} \\
 \hline \hline
Code for collection of ADS-B data & Nov 28, 2023 & Dec 1, 2023 \\
\hline
First Week ADS-B data collection & Dec 1, 2023 & Dec 7, 2023 \\
\hline
First Sample Collection & Dec 7 2023 & Dec 7, 2023 \\
\hline
First batch of samples lab work & Dec 7, 2023 & Dec 8, 2023 \\ 
\hline
Second Week ADS-B data collection & Dec 8, 2023 & Dec 14, 2023 \\
\hline
Second Sample Collection & Dec 14 2023 & Dec 14, 2023 \\
\hline
First batch of samples lab work & Dec 14, 2023 & Dec 15, 2023 \\ 
\hline
Repeat for 3rd and 4th week & Dec 15, 2023 & Dec 29, 2023\\
 \hline
Data processing for first month& Jan 10, 2024& Jan 29, 2024\\
 \hline
 Repeat for 5th-8th week\footnote{Since in winter the GA trafic is lower, we wait till the weather become warmer to collect more samples} & March 1, 2024 & March 29, 2024\\
  \hline
All data processing and finishing the project& April 1, 2024 & April 29, 2024\\
\hline
\end{tabular}
\end{center}
\section{Conclusions}
\section{Budget and Justification}

\newpage
\printbibliography
\end{document}
